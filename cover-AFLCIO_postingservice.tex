% !TeX root = ./cv.tex

\title{Cover Letter}

\maketitle

\newcommand{\employer}{the AFL-CIO}

%\noindent\textsc{To:}\\
\toaddress
	{AFL-CIO}
	{Resume Posting Service}
	{815 16th St NW}
	{Washington}
	{DC}
	{20006}
	{United States}
	
I would like to express interest in joining the labor movement as
an organizer or strategic researcher. As an Anthropology graduate from
the University of Illinois with strong research and data manipulation
skills, I am certain I can be a welcome candidate for this position.
Studying the field of Anthropology has given me 
social and research skills that
are particularly attuned to the needs of \employer.
The discipline is heavily focused on understanding peoples of disparate
backgrounds and cultures, which influences not only my work but my worldview and
ideology. Furthermore, I have shown a diligent work ethic throughout
my post-collegiate employment.

My research skill includes great
experience in both qualitative and quantitative data interpretation: in 2017
I conducted an independent research project collecting data on wild
capuchin monkeys for several weeks, performing statistical hypothesis
tests on the data I collected; in 2018, I spent a semester in Kenya, where
I surveyed a population of persons with disabilities and collected qualitative
data on their life experiences. My final project involved a comprehensive
and quantitative literature review on published ecological data.
In whatever campaign the union embarks on, I can provide
diligent and professional research skill.

As a student of Anthropology, I have spent extended time in different
cultures as part of my studies. Understanding how people live and work
is an integral part of the discipline. During my research project in Kenya,
I studied how the segregated workshop model, in which
a workplace almost exclusively hires persons with disabilities
and pays them below minimum wage, fails to live up to the utopian dream
it is predicated on and actively exploits precarious workers.
In the case of the workplace I studied, this exploitation was
exacerbated by the employer also providing housing for the workers,
acting as both landlord and boss.
Throughout this research project, I spoke to and worked
with the workers I studied, ingratiating myself as both
researcher and ally. While I don't expect organizers to
have explicit sanction to intrude on and directly talk
to workers at the workplace, my abilities to befriend
and learn from workers can be a valuable asset to a union.
My goal to work in the labor movement stems from this
research, giving me a strong desire to
help similarly precarious and oft-forgotten workers,
from persons with disabilities
to undocumented workers or the new masses of independent contractors.

In my current job, I work 84 hours per week in \textsc{Covid-19} relief,
decontaminating N95 masks for FEMA. The position is 100\% travel,
so I can commit to working long hours or significant travel.
In my position, I have continuously taken the initiative to monitor
for safety issues at the worksite, from simple electrical issues to
PPE failures. While this is more salient at a job working directly
with contaminated medical waste, safety is an important part of the
employment experience for all workers. My experience monitoring for
and understanding workplace safety can help the labor movement
protect working people.

I hope you are able to find that my skills and drive are well-attuned
to assisting the labor movement.

\signature