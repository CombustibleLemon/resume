% !TeX root = ./cv.tex
\title{Cover Letter}
\newgeometry{top=1in,bottom=1in,left=0.8in,right=0.8in}
% \vspace*{-0.2in}
\maketitle

\begin{adjustwidth}{0.2in}{0.2in}

%\noindent\textsc{To:}\\
\toaddress
    {Strategic Corporate Research Summer School}
    {262 Ives Faculty Building}
    {Cornell University}
    {Ithaca}
    {NY}
    {14853}
    {United States}

\noindent
In my collegiate career, I have developed a passionate interest in the labor movement,
particularly from a comparative international point of view.
As a recent graduate of anthropology at the University of Illinois,
I have developed skills in both research and cross-cultural comprehension that are
integral to both my worldview and my career goals. The discipline of anthropology
is heavily focused on understanding peoples of disparate backgrounds and cultures,
while simultaneously teasing out the social rules that govern behavior, whether they
stem from gender, class, or labor relations.

My interests in understanding international differences in labor relations
stem directly from my anthropological studies.
In the fall of 2018, I had the wonderful opportunity to spend the semester abroad
in Kenya, where I could be exposed to a radically different culture from my own
and which allowed me to practice my training as both an anthropologist and as a researcher.
For six weeks, I conducted ethnographic research on a population of persons with
disabilities who had been terminated from a segregated workshop following budget cuts.
What drew me to this population was the exceptionally strong power dynamic between the
employer and the worker: nearly all workers lived in housing owned by the company,
paying for their rent via docked pay. With termination thus necessarily came eviction,
an exceptionally cruel blow to a precariat in a country where the built environment is
exceptionally harsh toward persons with disabilities.

In researching the impact of segregated workshops on their labor force for this project,
I discovered areas of study and of the labor movement that had been previously unknown to me.
The intersections of disability, colonialism, poverty, and labor were deeply intertwined
for this population, and I was able to deeply explore the history of segregated labor for
the disabled and the impact of international commercial markets on the precariat of the Global South.

% My interests in the Strategic Corporate Research Summer School derive from my experience
% with the Kenyan workers. I seek to understand the relations corporations have with their
% precariat laborers, whether they be Global South citizens, persons with disabilities, or
% the impoverished. 

Discovering the Kenyan workers' relationship to their employer extending into their housing,
I was immediately reminded of the American history of company towns. While the laborers I
worked with were not paid in scrip as in Pullman or nineteenth-century mining towns,
the connection between work and home life was regardless extraordinarily deep.
I was most surprised, however, to learn that the practice is becoming once again fashionable
in large American corporations. Silicon Valley companies have been lauded in the business
press during the past half-decade by providing housing benefits to their employees,
whether they be a bonus to live as close as possible to work. %~\autocite{Oran2015}
or prefabricated housing units for employees. %~\autocite{Kusisto2017}.
Facebook has published plans to construct an entire company town,
containing not only employee housing but amenities such as a grocery store and
hotel. %~\autocite{Hartmans2017}.

In a more extreme example, some governments have not simply permitted corporations
to construct large campuses but actively encouraged the resurgence of company towns.
During Amazon's pursuit of a second headquarters location, Kankakee County, Illinois
offered the trillion-dollar company the right to form a municipal government in
unincorporated land, with utilities and even schools under the management of
Amazon. %~\autocite{Lauterbach2017}.
More recently, the governor of Nevada has proposed a similar deal, allowing any
company to acquire and control swaths of the state as incorporated towns. %~\autocite{Lochhead2021}.
The labor movement has a history of interaction with company towns, from violent fights at 
early mining towns to the great Pullman Strike in 1894. 
Understanding the impact of employer-owned housing, from both a historical perspective and
through modern strategic corporate research, will likely come to be imperative
for the labor movement if it seeks to grow in the traditionally non-unionized
technology professionals' industry.

As an anthropologist, no discussion of a social structure would
be complete without mention of culture. I gained great interest in
the modern corporate structure after reading \citefirstlastauthor{Orta2019}'s
intriguing analysis of contemporary business school culture,
\citetitle{Orta2019}.

My experience in academic research extends beyond the qualitative ethnographic work in Kenya.
In my sophomore year I conducted an independent research project observing wild
capuchin monkeys for several weeks, performing statistical hypothesis testing on the data I collected;
my most recent project involved a comprehensive and quantitative
literature review on published ecological data.
As a result of this research experience, I have exceptional communication and writing skills.
For all of my undergraduate research projects, I developed presentations,
including posters and slideshows, focused both for colleagues and laypersons.
With this comprehensive experience, I am certain I am prepared to take on the
requirements of the advanced track of the program.

% I am also proficient with technology, an integral part of the program and of
% contemporary research methodology. Growing up at the turn of the century,
% I rapidly learned the ins and outs of contemporary computers, from their physical
% structure to the everyday software they run. For statistical analysis,
% I have knowledge in R and Python; in document preparation I am deeply familiar with \LaTeX\ 
% and Microsoft Word; finally, I can use Microsoft Excel very well for data analysis and visualization.
% I have great experience using the internet for research, whether as a medium in and of itself
% or as a tool for finding sources in academic databases or library archives.
% I have no doubt that my computer skill will be lacking in the areas needed by the summer school program.

Following completion of the summer school, I intend to continue into a career as a
researcher within the labor movement. As might be implied by my previous experience,
I am especially interested in unions working amongst precariat and very low-wage workers.
Most recently, the Fight for \$15 movement has captivated me in its uniting workers who are
disproportionately precarious and impoverished. Having gained experience from Cornell,
I can take my research expertise to task for the labor movement, assisting unions in
gaining fair contracts for low-wage precariat. My understanding of disability rights
in the workplace can provide an added benefit to especially the Fight for \$15 movement,
as the already-low minimum wage at present provides exemptions for employers who hire
persons with disabilities, %~\autocite{NDRN2011,ESA1991},
a fact often overlooked in even the most vociferous minimum-wage organizing drives.

Ultimately, I hope my participation in the ILR's program will allow me to be a small
but nonetheless important part of the advancement of labor in twenty-first century America.
Combining my anthropological education, research experience, and drive to improve
the lives of workers, I intend to be an effective and diligent corporate researcher
for the labor movement.

\signature

% \nocite{Orta2019}\printbibliography

\end{adjustwidth}
\restoregeometry
